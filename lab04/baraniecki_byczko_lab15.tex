%!TEX program = xelatex
%Template created by: Maciej Byczko
\documentclass[a4paper,12pt]{extarticle}  %typ dokumentu

\usepackage{geometry} %poprawienie marginesów
\usepackage{polski} %polskie znaki
\usepackage{graphicx} %grafiki
\usepackage{float} %poprawienie pozycji
\usepackage{fancyhdr} % header i footer
\usepackage{listings}
\usepackage{xcolor}
\usepackage{hyperref}
\graphicspath{{pictures/}}
\geometry{margin=0.7in}
\pagestyle{fancy}
\cfoot{Strona \thepage}
\rhead{Strona \thepage}
\lhead{\typdoc}
\setlength{\headheight}{15pt}

\title{\tytul \\ \small{\opis}}
\author{\tworcy}
\date{\data}

%-----------------------SEKCJA DANYCH----------------------------------
\def\tytul{Sterowaniem silnikiem krokowym za pomocą USB} %<<< tytuł ćwiczenia
\def\nrcw{laboratoria 15} %<<< numer ćwiczenia
\def\data{\today} %<< data wykonania
\def\prowadzacy{Dr inż. Dominik Żelazny} %<<<prowadzący
\def\nrgrupy{D} %<<<numer grupy
\def\tworcy{Baraniecki Karol\\Byczko Maciej} %<<< autorzy
\def\zajinfo{PT 16:30 TP} %<<< informacje dotyczące zajęć
\def\typdoc{Sprawozdanie} %<<< typ dokumentu tj Sprawozdanie, zadania itp. {Matematyka dyskretna/Sprawozdanie z Miernictwa}
\def\opis{} %<<< opis który będzie umieszczony pod tytułem w Maketitle
%----------------------------------------------------------------------

\definecolor{backcolour}{rgb}{0.95,0.95,0.92}
\definecolor{AO}{rgb}{0,0.5,0}
\definecolor{ZeroBlue}{rgb}{0,0.28,0.73}
\definecolor{DarkRed}{rgb}{0.85,0.16,0.16}


\lstset{
basicstyle=\footnotesize,
breaklines=true,
language=Python,
numbers=left,
tabsize=2,
numberstyle=\tiny,
backgroundcolor=\color{backcolour},
breakatwhitespace=false,
showspaces=false,                
showstringspaces=false,
showtabs=false,
commentstyle=\color{gray},
keywordstyle=\color{ZeroBlue},
keepspaces=true,
% keywordstyle={[2]\color{DarkRed}},
% keywordstyle={[3]\color{ZeroBlue}},
}

\begin{document}
%-------------------------------------TABELA-DANYCH--------------------------------------------------
\begin{table}[H]
	\centering
	\resizebox{\textwidth}{!}{
		\begin{tabular}{|c|c|c|}\hline
			\begin{tabular}[c]{@{}c@{}}                     \tworcy     \end{tabular} &
			\begin{tabular}[c]{@{}c@{}}Prowadzący:\\        \prowadzacy \end{tabular} &
			\begin{tabular}[c]{@{}c@{}}Numer ćwiczenia\\    \nrcw       \end{tabular}          \\ \hline
			\begin{tabular}[c]{@{}c@{}}                     \zajinfo    \end{tabular} &
			\begin{tabular}[c]{@{}c@{}}Temat ćwiczenia:\\   \tytul      \end{tabular} & Ocena: \\ \hline
			\begin{tabular}[c]{@{}c@{}}Grupa:\\          \nrgrupy    \end{tabular}    &
			\begin{tabular}[c]{@{}c@{}}Data wykonania:\\    \data       \end{tabular} &        \\ \hline
		\end{tabular}}
\end{table}
%----------------------------------------------------------------------------------------------------
\section{Zagadnienia do opracowania}
\begin{enumerate}
	\item Obsługa API Windows, dołączanie bibliotek w formatach COFF i OMF
	\item Budowa, zasadza działania oraz sterowanie silnikiem krokowym.
	\item Zapoznać się z budową i działaniem układu FT245BM oraz modułem zawierającym ten układ Mmusb245 a także ze schematem układu makiety zamieszczonym poniżej.
	\item Zapoznać się z instrukcją biblioteki FT2XX.dll oraz programatora FTD2XXST.EXE.
\end{enumerate}
\section{Zadania do wykonania}
\begin{enumerate}
	\item Zapoznaj się z właściwościami systemowymi urządzenia USB sterującego kamerą,
	      \begin{itemize}
		      \item (panel sterowania oraz program FTD2XXST.EXE - opcja test. (zwróć uwagę na Vendor ID (VID), Product ID (PID), Description)
	      \end{itemize}

	\item Uruchom przykładowy program do obsługi sterownika i przetestuj jego możliwości.
	      \begin{itemize}
		      \item zaprogramuj układ
		      \item zresetuj układ
		      \item zapoznaj się z działaniem jego pozostałych funkcji
		      \item zbadaj dla jakich parametrów układ działa poprawnie
		      \item zbadaj jaki wpływ na działanie aplikacji oraz rozpoznawanie urządzenia przez system ma zaprogramowanie oraz resetowanie układu.
	      \end{itemize}

	\item Do Borland'a C++ Buildera 6.0  skonwertuje bibliotekę FT2XX.dll do postaci OMF. (oryginalny plik FT2XX.lib jest w postaci COFF i nie jest obsługiwany przez ta wersje kompilatora )
	\item Właściwe ustawienia urządzanie (proszę ich nie przeprogramowywać):
	      \begin{itemize}
		      \item Vendor ID = 0403 (VID),
		      \item Product ID =6001 (PID),
		      \item Device Description ="usb step motor"
	      \end{itemize}

	\item Napisz aplikacje umożliwiającą obracanie kamerą w poziomie.
	\item (Uwaga utrzymywanie przez dłuższy czas zasilania któregokolwiek z uzwojeń silnika może spowodować jego uszkodzenie - proszę pamiętać ze silnik krokowy sterowany jest impulsami, w schemacie jest błąd - dwa piny są zamienione!)
	\item Wzbogać aplikację o możliwość obracania kamerą w pionie.
	\item Wzbogać aplikację o możliwość zadawania prędkość obrotu.
	\item Wzbogać aplikacje o możliwość zadawania ilość kroków.
	\item Zaproponuj rozwiązanie podłączania wielu urządzeń ze sterownikiem FT245BM do jednego komputera. (dokonaj odpowiednich zmian w kodzie programu).
	\item Wzbogać obie napisane aplikacje o kontrolę błędów i komunikację z użytkownikiem
	\item Zresetuj układ do ustawień standardowych.
\end{enumerate}
\section{Wnioski}
\end{document}