%!TEX program = xelatex
%Template created by: Maciej Byczko
\documentclass[a4paper,12pt]{extarticle}  %typ dokumentu

\usepackage{geometry} %poprawienie marginesów
\usepackage{polski} %polskie znaki
\usepackage{graphicx} %grafiki
\usepackage{float} %poprawienie pozycji
\usepackage{fancyhdr} % header i footer
\usepackage{listings}
\usepackage{xcolor}
\usepackage{hyperref}
\graphicspath{{pictures/}}
\geometry{margin=0.7in}
\pagestyle{fancy}
\cfoot{Strona \thepage}
\rhead{Strona \thepage}
\lhead{\typdoc}
\setlength{\headheight}{15pt}

\title{\tytul \\ \small{\opis}}
\author{\tworcy}
\date{\data}

%-----------------------SEKCJA DANYCH----------------------------------
\def\tytul{Obsługa karty muzycznej z wykorzystaniem DirectSound, API i ActiveX} %<<< tytuł ćwiczenia
\def\nrcw{laboratoria 11} %<<< numer ćwiczenia
\def\data{\today} %<< data wykonania
\def\prowadzacy{Dr inż. Dominik Żelazny} %<<<prowadzący
\def\nrgrupy{D} %<<<numer grupy
\def\tworcy{Baraniecki Karol\\Byczko Maciej} %<<< autorzy
\def\zajinfo{PT 16:30 TP} %<<< informacje dotyczące zajęć
\def\typdoc{Sprawozdanie} %<<< typ dokumentu tj Sprawozdanie, zadania itp. {Matematyka dyskretna/Sprawozdanie z Miernictwa}
\def\opis{} %<<< opis który będzie umieszczony pod tytułem w Maketitle
%----------------------------------------------------------------------

\definecolor{backcolour}{rgb}{0.95,0.95,0.92}
\definecolor{AO}{rgb}{0,0.5,0}
\definecolor{ZeroBlue}{rgb}{0,0.28,0.73}
\definecolor{DarkRed}{rgb}{0.85,0.16,0.16}


\lstset{
basicstyle=\footnotesize,
breaklines=true,
language=Python,
numbers=left,
tabsize=2,
numberstyle=\tiny,
backgroundcolor=\color{backcolour},
breakatwhitespace=false,
showspaces=false,                
showstringspaces=false,
showtabs=false,
commentstyle=\color{gray},
keywordstyle=\color{ZeroBlue},
keepspaces=true,
% keywordstyle={[2]\color{DarkRed}},
% keywordstyle={[3]\color{ZeroBlue}},
}

\begin{document}
%-------------------------------------TABELA-DANYCH--------------------------------------------------
\begin{table}[H]
	\centering
	\resizebox{\textwidth}{!}{
		\begin{tabular}{|c|c|c|}\hline
			\begin{tabular}[c]{@{}c@{}}                     \tworcy     \end{tabular} &
			\begin{tabular}[c]{@{}c@{}}Prowadzący:\\        \prowadzacy \end{tabular} &
			\begin{tabular}[c]{@{}c@{}}Numer ćwiczenia\\    \nrcw       \end{tabular}          \\ \hline
			\begin{tabular}[c]{@{}c@{}}                     \zajinfo    \end{tabular} &
			\begin{tabular}[c]{@{}c@{}}Temat ćwiczenia:\\   \tytul      \end{tabular} & Ocena: \\ \hline
			\begin{tabular}[c]{@{}c@{}}Grupa:\\          \nrgrupy    \end{tabular}    &
			\begin{tabular}[c]{@{}c@{}}Data wykonania:\\    \data       \end{tabular} &        \\ \hline
		\end{tabular}}
\end{table}
%----------------------------------------------------------------------------------------------------
\section{Zagadnienia do opracowania}
\begin{enumerate}
	\item Prosty mechanizm odtwarzaniu dźwięku:
	      \begin{itemize}
		      \item Komenda PlaySound()
		      \item Komponent ActiveX 'Windows Media Player':
	      \end{itemize}
	\item Plik WAV
	      \begin{itemize}
		      \item Rodzaje i formaty nagłówka pliku WAV
		      \item Informacje przechowywane w nagłówku pliku WAV
	      \end{itemize}
	\item Waveform - Komendy typu WaveOut* potrzebne do odtwarzania dźwięku
	\item DirectSound - DirectX 10.0 Component DirectSound
\end{enumerate}
\section{Zadania do wykonania}
\begin{enumerate}
	\item Korzystając z przykładowej aplikacji odtworzyć dźwięk za pomocą trzech sposobów: ActiveX,Waveform and Auxiliary Audio i DirectSound.
	\item Napisać program, odtwarzający dźwięk z wykorzystaniem komendy PlaySound() oraz z wykorzystaniem ActiveX`a.
	\item Napisać program sczytujący nagłówek WAVa i wyświetlić poszczególne wartości.
	\item Napisać program odtwarzający WAVa za pomocą DirectSound i Waveform and Auxiliary Audio z wykorzystaniem komendy waveOutWrite().
	\item Napisać program odtwarzający wybrany (z listy plików) plik w formacie MP3 (wykorzystać WindowsMediaPlayer)
	\item Napisać aplikację która sczytuje  z mikrofonu dane i zapisuje je do pliku.
	\item Zadania uzupełniające (alternatywnie do wyboru):
	      \begin{itemize}
		      \item uzupełnić aplikację odtwarzającą dźwięk za pomocą DirectSound o dodatkowe efekty dźwiękowe (np. panning) - wykorzystać implementację efektów z DirectSound-a
		      \item wyświetlić na ekranie moc sygnału (np. słupki) dla kilku przedziałów częstotliwości
	      \end{itemize}
\end{enumerate}
\section{Wnioski}
\end{document}