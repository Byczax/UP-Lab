%!TEX program = xelatex
%Template created by: Maciej Byczko
\documentclass[a4paper,12pt]{extarticle}  %typ dokumentu

\usepackage{geometry} %poprawienie marginesów
\usepackage{polski} %polskie znaki
\usepackage{graphicx} %grafiki
\usepackage{float} %poprawienie pozycji
\usepackage{fancyhdr} % header i footer
\usepackage{listings}
\usepackage{xcolor}
\graphicspath{{pictures/}}
\geometry{margin=0.7in}
\pagestyle{fancy}
\cfoot{Strona \thepage}
\rhead{Strona \thepage}
\lhead{\typdoc}
\setlength{\headheight}{15pt}

\title{\tytul \\ \small{\opis}}
\author{\tworcy}
\date{\data}

%-----------------------SEKCJA DANYCH----------------------------------
\def\tytul{USB, DirectInput} %<<< tytuł ćwiczenia
\def\nrcw{laboratoria 10} %<<< numer ćwiczenia
\def\data{\today} %<< data wykonania
\def\prowadzacy{Dr inż. Dominik Żelazny} %<<<prowadzący
\def\nrgrupy{D} %<<<numer grupy
\def\tworcy{Baraniecki Karol\\Byczko Maciej} %<<< autorzy
\def\zajinfo{PT 16:30 TP} %<<< informacje dotyczące zajęć
\def\typdoc{Sprawozdanie} %<<< typ dokumentu tj Sprawozdanie, zadania itp. {Matematyka dyskretna/Sprawozdanie z Miernictwa}
\def\opis{} %<<< opis który będzie umieszczony pod tytułem w Maketitle
%----------------------------------------------------------------------

\definecolor{backcolour}{rgb}{0.95,0.95,0.92}
\definecolor{AO}{rgb}{0,0.5,0}
\definecolor{ZeroBlue}{rgb}{0,0.28,0.73}
\definecolor{DarkRed}{rgb}{0.85,0.16,0.16}


\lstset{
breaklines=true,
language=Python,
numbers=left,
tabsize=2,
numberstyle=\tiny,
backgroundcolor=\color{backcolour},
breakatwhitespace=false,
showspaces=false,                
showstringspaces=false,
showtabs=false,
commentstyle=\color{gray},
keywordstyle=\color{ZeroBlue},
% keywordstyle={[2]\color{DarkRed}},
% keywordstyle={[3]\color{ZeroBlue}},
}

\begin{document}
%-------------------------------------TABELA-DANYCH--------------------------------------------------
\begin{table}[H]
	\centering
	\resizebox{\textwidth}{!}{
		\begin{tabular}{|c|c|c|}\hline
			\begin{tabular}[c]{@{}c@{}}                     \tworcy     \end{tabular} &
			\begin{tabular}[c]{@{}c@{}}Prowadzący:\\        \prowadzacy \end{tabular} &
			\begin{tabular}[c]{@{}c@{}}Numer ćwiczenia\\    \nrcw       \end{tabular}          \\ \hline
			\begin{tabular}[c]{@{}c@{}}                     \zajinfo    \end{tabular} &
			\begin{tabular}[c]{@{}c@{}}Temat ćwiczenia:\\   \tytul      \end{tabular} & Ocena: \\ \hline
			\begin{tabular}[c]{@{}c@{}}Grupa:\\          \nrgrupy    \end{tabular} &
			\begin{tabular}[c]{@{}c@{}}Data wykonania:\\    \data       \end{tabular} &        \\ \hline
		\end{tabular}}
\end{table}
%----------------------------------------------------------------------------------------------------
\section{Zadania do opracowania}
Teoretyczny opis elementów używanych do bezpośredniej kontroli (Direct Input)
\subsection{USB w Windows}
\subsubsection{Warstwa HAL i HEL}
HAL - Hardware Abstraction Layer, "nieprzenośna" część jądra NT.\\
HEL - Hardware Emulation Library, jak sama nazwa mówi, służy do emulacji urządzeń, problem ze znalezieniem dokumentacji, raczej już nie używana.
\subsubsection{Standard USB oraz USB 2.0}
USB - Universal Serial Bus, po polsku Uniwersalna Magistrala Szeregowa, uniwersalny interfejs komunikacyjny typu plug and play.
Obecnie najnowszą wersją jest USB 3.2 wyprodukowany w 2017 roku.
\section{Zadania do wykonania}
\begin{enumerate}
	\item Napisać program, odczytujący nazwę zainstalowanego joysticka
	\item Napisać program ilustrujący działanie joysticka: stwierdzający naciśnięcie myszy, zmianę położenia drążka (w przestrzeni 2D) oraz suwaka.
	\item Napisać program zastępujący działanie myszy. Program ma umożliwiać sterowanie kursorem za pomocą joysticka oraz obsługę przycisków fire jako kliknięć myszy.
	\item Napisać program realizujący prosty edytor graficzny - rysowanie przy pomocy Joysticka
\end{enumerate}
\subsection{Odczytywanie nazwy joysticka}
\begin{lstlisting}
pygame.joystick.init()
pad = pygame.joystick.Joystick(0)
pad_name = pad.get_name()
\end{lstlisting}
\subsection{Kontrola myszki za pomocą joysticka}
Do wykonania tego zadania wykorzystaliśmy biblioteki \emph{Mouse} oraz \emph{PyGame}
\lstinputlisting{mouse_joystick_control.py}
\cleardoublepage
\subsection{Paint}
\subsection{odczytanie nazwy zainstalowanego joysticka}
\lstinputlisting{direct.py}
\subsubsection{Dowód działania programu graficznego}
\begin{figure}[H]
   \centering
   \resizebox*{\textwidth}{!}{
	  \includegraphics{program.png}
   }
\end{figure}
\end{document}