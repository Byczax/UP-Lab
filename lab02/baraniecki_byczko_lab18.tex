%!TEX program = xelatex
%Template created by: Maciej Byczko
\documentclass[a4paper,12pt]{extarticle}  %typ dokumentu

\usepackage{geometry} %poprawienie marginesów
\usepackage{polski} %polskie znaki
\usepackage{graphicx} %grafiki
\usepackage{float} %poprawienie pozycji
\usepackage{fancyhdr} % header i footer
\usepackage{listings}
\usepackage{xcolor}
\graphicspath{{pictures/}}
\geometry{margin=0.7in}
\pagestyle{fancy}
\cfoot{Strona \thepage}
\rhead{Strona \thepage}
\lhead{\typdoc}
\setlength{\headheight}{15pt}

\title{\tytul \\ \small{\opis}}
\author{\tworcy}
\date{\data}

%-----------------------SEKCJA DANYCH----------------------------------
\def\tytul{Analizator parametrów sieci - EMA-90N} %<<< tytuł ćwiczenia
\def\nrcw{laboratoria 18} %<<< numer ćwiczenia
\def\data{\today} %<< data wykonania
\def\prowadzacy{Dr inż. Dominik Żelazny} %<<<prowadzący
\def\nrgrupy{D} %<<<numer grupy
\def\tworcy{Baraniecki Karol\\Byczko Maciej} %<<< autorzy
\def\zajinfo{PT 16:30 TP} %<<< informacje dotyczące zajęć
\def\typdoc{Sprawozdanie} %<<< typ dokumentu tj Sprawozdanie, zadania itp. {Matematyka dyskretna/Sprawozdanie z Miernictwa}
\def\opis{} %<<< opis który będzie umieszczony pod tytułem w Maketitle
%----------------------------------------------------------------------

\definecolor{backcolour}{rgb}{0.95,0.95,0.92}
\definecolor{AO}{rgb}{0,0.5,0}
\definecolor{ZeroBlue}{rgb}{0,0.28,0.73}
\definecolor{DarkRed}{rgb}{0.85,0.16,0.16}


\lstset{
breaklines=true,
language=Python,
numbers=left,
tabsize=2,
numberstyle=\tiny,
backgroundcolor=\color{backcolour},
breakatwhitespace=false,
showspaces=false,                
showstringspaces=false,
showtabs=false,
commentstyle=\color{gray},
keywordstyle=\color{ZeroBlue},
% keywordstyle={[2]\color{DarkRed}},
% keywordstyle={[3]\color{ZeroBlue}},
}

\begin{document}
%-------------------------------------TABELA-DANYCH--------------------------------------------------
\begin{table}[H]
	\centering
	\resizebox{\textwidth}{!}{
		\begin{tabular}{|c|c|c|}\hline
			\begin{tabular}[c]{@{}c@{}}                     \tworcy     \end{tabular} &
			\begin{tabular}[c]{@{}c@{}}Prowadzący:\\        \prowadzacy \end{tabular} &
			\begin{tabular}[c]{@{}c@{}}Numer ćwiczenia\\    \nrcw       \end{tabular}          \\ \hline
			\begin{tabular}[c]{@{}c@{}}                     \zajinfo    \end{tabular} &
			\begin{tabular}[c]{@{}c@{}}Temat ćwiczenia:\\   \tytul      \end{tabular} & Ocena: \\ \hline
			\begin{tabular}[c]{@{}c@{}}Grupa:\\          \nrgrupy    \end{tabular} &
			\begin{tabular}[c]{@{}c@{}}Data wykonania:\\    \data       \end{tabular} &        \\ \hline
		\end{tabular}}
\end{table}
%----------------------------------------------------------------------------------------------------
\section{Zadania do opracowania}
\subsection{Sieć elektryczna}
\begin{itemize}
	\item napięcie - różnica potencjałów elektrycznych między dwoma punktami obwodu elektrycznego lub pola elektrycznego.
	\item prąd - uporządkowany ruch ładunków elektrycznych
	\item moc czynna - część mocy, którą odbiornik pobiera ze źródła i zamienia na pracę lub ciepło.
	\item moc bierna - wielkość opisująca pulsowanie energii elektrycznej między elementami obwodu elektrycznego.
	\item cos($\phi$) - Współczynnik mocy, stosunek mocy czynnej do mocy pozornej, czyli stosunek mocy użytecznej do iloczynu napięcia i prądu. 
	% \item harmoniczna prądu
\end{itemize}
\subsection{Ethernet}
\begin{itemize}
	\item IP (Internet Protocol) - protokół komunikacyjny warstwy sieciowej modelu OSI (warstwy internetu w modelu TCP/IP).
	\item Maska - liczba służąca do wyodrębnienia w adresie IP części będącej adresem podsieci i części, która jest adresem hosta w tej podsieci. 
	\item Brama domyślna - router, do którego komputery sieci lokalnej mają wysyłać pakiety o ile nie powinny być one kierowane w sieć lokalną lub do innych, znanych im routerów.
	\item DHCP (Dynamic Host Configuration Protocol) - protokół komunikacyjny umożliwiający hostom uzyskanie od serwera danych konfiguracyjnych, np. adresu IP hosta, adresu IP bramy sieciowej, adresu serwera DNS, maski podsieci.
\end{itemize}
\subsection{Protokół modbus TCP/IP}
Modbus to popularny protokół komunikacyjny w którym komunikacja między urządzeniami realizowana jest w architekturze master-slave/client-server. 
Jest to protokół typu otwartego, co oznacza iż wszystkie niezbędne informacje do jego implementacji są ogólnodostępne.
% \subsection{Bezpieczeństwo pracy z prądem}

\section{Zadania do wykonania}
Połączyć urządzenie EMA-90N z komputerem za pomocą komunikacji Ethernet
Uruchomić aplikację demonstracyjną i połączyć się z urządzaniem odczytując napięcie i prąd na L1
Napisać aplikację w C#, która połączy się z urządzeniem i umożliwi odczytanie napięcia i prądu L1 z użyciem protokołu modbus.
\end{document}