%!TEX program = xelatex
%Template created by: Maciej Byczko
\documentclass[a4paper,12pt]{extarticle}  %typ dokumentu

\usepackage{geometry} %poprawienie marginesów
\usepackage{polski} %polskie znaki
\usepackage{graphicx} %grafiki
\usepackage{float} %poprawienie pozycji
\usepackage{fancyhdr} % header i footer
\usepackage{listings}
\usepackage{xcolor}
\usepackage{hyperref}
\graphicspath{{pictures/}}
\geometry{margin=0.7in}
\pagestyle{fancy}
\cfoot{Strona \thepage}
\rhead{Strona \thepage}
\lhead{\typdoc}
\setlength{\headheight}{15pt}

\title{\tytul \\ \small{\opis}}
\author{\tworcy}
\date{\data}

%-----------------------SEKCJA DANYCH----------------------------------
\def\tytul{GPS} %<<< tytuł ćwiczenia
\def\nrcw{laboratoria 7} %<<< numer ćwiczenia
\def\data{\today} %<< data wykonania
\def\prowadzacy{Dr inż. Dominik Żelazny} %<<<prowadzący
\def\nrgrupy{D} %<<<numer grupy
\def\tworcy{Baraniecki Karol\\Byczko Maciej} %<<< autorzy
\def\zajinfo{PT 16:30 TP} %<<< informacje dotyczące zajęć
\def\typdoc{Sprawozdanie} %<<< typ dokumentu tj Sprawozdanie, zadania itp. {Matematyka dyskretna/Sprawozdanie z Miernictwa}
\def\opis{} %<<< opis który będzie umieszczony pod tytułem w Maketitle
%----------------------------------------------------------------------

\definecolor{backcolour}{rgb}{0.95,0.95,0.92}
\definecolor{AO}{rgb}{0,0.5,0}
\definecolor{ZeroBlue}{rgb}{0,0.28,0.73}
\definecolor{DarkRed}{rgb}{0.85,0.16,0.16}


\lstset{
basicstyle=\footnotesize,
breaklines=true,
language=Python,
numbers=left,
tabsize=2,
numberstyle=\tiny,
backgroundcolor=\color{backcolour},
breakatwhitespace=false,
showspaces=false,                
showstringspaces=false,
showtabs=false,
commentstyle=\color{gray},
keywordstyle=\color{ZeroBlue},
keepspaces=true,
% keywordstyle={[2]\color{DarkRed}},
% keywordstyle={[3]\color{ZeroBlue}},
}

\begin{document}
%-------------------------------------TABELA-DANYCH--------------------------------------------------
\begin{table}[H]
	\centering
	\resizebox{\textwidth}{!}{
		\begin{tabular}{|c|c|c|}\hline
			\begin{tabular}[c]{@{}c@{}}                     \tworcy     \end{tabular} &
			\begin{tabular}[c]{@{}c@{}}Prowadzący:\\        \prowadzacy \end{tabular} &
			\begin{tabular}[c]{@{}c@{}}Numer ćwiczenia\\    \nrcw       \end{tabular}          \\ \hline
			\begin{tabular}[c]{@{}c@{}}                     \zajinfo    \end{tabular} &
			\begin{tabular}[c]{@{}c@{}}Temat ćwiczenia:\\   \tytul      \end{tabular} & Ocena: \\ \hline
			\begin{tabular}[c]{@{}c@{}}Grupa:\\          \nrgrupy    \end{tabular}    &
			\begin{tabular}[c]{@{}c@{}}Data wykonania:\\    \data       \end{tabular} &        \\ \hline
		\end{tabular}}
\end{table}
%----------------------------------------------------------------------------------------------------
\section{Zagadnienia do opracowania}
\begin{enumerate}
	\item Zasady działania systemu GPS
	\item Przegląd typów urządzeń GPS oraz sposobów łączności z komputerem
	\item Zaposznać się z dokumentacja do urządzenia GPS firmy Nokia, model LD-1W
	\item Protokół NMEA
	\item Obsługa transmisji szeregowej w środowiskach obiektowych (połączenie Bluetooth i port szeregowy emulowany)
	\item Dostęp do map z poziomu aplikacji (np. Google maps)
\end{enumerate}
\section{Zadania do wykonania}
\begin{enumerate}
	\item Zapoznać się z zestawem GPS oraz podłączyć via Bluetooth
	\item W ramach testu podłączyć GPS na ustawieniach testowych oraz utworzyć połączenie przy użyciu HyperTerminala.
	\item Odczytać uzyskane komendy oraz podzielić je wg typów wiadomości.
	\item Sprawdzić ważność uzyskanych danych i przedyskutować wynik.
	\item Napisać program w dowolnym środowisku obiektowym, który będzie obsługiwał transmisję szeregową oraz pozwoli na czytelne przedstawienie uzyskanych danych.
	\item Napisać program, który na podstawie samodzielnie uzyskanych danych lub od prowadzącego (plik tekstowy, format NMEA) zlokalizuje na mapie świata (np. z Google Map) punkty, w których znajdowało się urządzenie.
\end{enumerate}
\section{Wnioski}
\end{document}