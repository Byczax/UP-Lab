%!TEX program = xelatex
%Template created by: Maciej Byczko
\documentclass[a4paper,12pt]{extarticle}  %typ dokumentu

\usepackage{geometry} %poprawienie marginesów
\usepackage{polski} %polskie znaki
\usepackage{graphicx} %grafiki
\usepackage{float} %poprawienie pozycji
\usepackage{fancyhdr} % header i footer
\usepackage{listings}
\usepackage{xcolor}
\graphicspath{{pictures/}}
\geometry{margin=0.7in}
\pagestyle{fancy}
\cfoot{Strona \thepage}
\rhead{Strona \thepage}
\lhead{\typdoc}
\setlength{\headheight}{15pt}

\title{\tytul \\ \small{\opis}}
\author{\tworcy}
\date{\data}

%-----------------------SEKCJA DANYCH----------------------------------
\def\tytul{Bluetooth - komunikacja z telefonem komórkowym} %<<< tytuł ćwiczenia
\def\nrcw{laboratoria 14} %<<< numer ćwiczenia
\def\data{\today} %<< data wykonania
\def\prowadzacy{Dr inż. Dominik Żelazny} %<<<prowadzący
\def\nrgrupy{D} %<<<numer grupy
\def\tworcy{Baraniecki Karol\\Byczko Maciej} %<<< autorzy
\def\zajinfo{PT 16:30 TP} %<<< informacje dotyczące zajęć
\def\typdoc{Sprawozdanie} %<<< typ dokumentu tj Sprawozdanie, zadania itp. {Matematyka dyskretna/Sprawozdanie z Miernictwa}
\def\opis{} %<<< opis który będzie umieszczony pod tytułem w Maketitle
%----------------------------------------------------------------------

\definecolor{backcolour}{rgb}{0.95,0.95,0.92}
\definecolor{AO}{rgb}{0,0.5,0}
\definecolor{ZeroBlue}{rgb}{0,0.28,0.73}
\definecolor{DarkRed}{rgb}{0.85,0.16,0.16}


\lstset{
basicstyle=\footnotesize,
breaklines=true,
language=Python,
numbers=left,
tabsize=2,
numberstyle=\tiny,
backgroundcolor=\color{backcolour},
breakatwhitespace=false,
showspaces=false,                
showstringspaces=false,
showtabs=false,
commentstyle=\color{gray},
keywordstyle=\color{ZeroBlue},
keepspaces=true,
% keywordstyle={[2]\color{DarkRed}},
% keywordstyle={[3]\color{ZeroBlue}},
}

\begin{document}
%-------------------------------------TABELA-DANYCH--------------------------------------------------
\begin{table}[H]
	\centering
	\resizebox{\textwidth}{!}{
		\begin{tabular}{|c|c|c|}\hline
			\begin{tabular}[c]{@{}c@{}}                     \tworcy     \end{tabular} &
			\begin{tabular}[c]{@{}c@{}}Prowadzący:\\        \prowadzacy \end{tabular} &
			\begin{tabular}[c]{@{}c@{}}Numer ćwiczenia\\    \nrcw       \end{tabular}          \\ \hline
			\begin{tabular}[c]{@{}c@{}}                     \zajinfo    \end{tabular} &
			\begin{tabular}[c]{@{}c@{}}Temat ćwiczenia:\\   \tytul      \end{tabular} & Ocena: \\ \hline
			\begin{tabular}[c]{@{}c@{}}Grupa:\\          \nrgrupy    \end{tabular}    &
			\begin{tabular}[c]{@{}c@{}}Data wykonania:\\    \data       \end{tabular} &        \\ \hline
		\end{tabular}}
\end{table}
%----------------------------------------------------------------------------------------------------
\section{Zagadnienia do opracowania}
\subsection{Zasady korzystania z WinXP SP2 Api (nie z portu szeregowego !!!!!!)}
\subsection{MS Platform SDK}
\subsection{Znajomość funkcji}
Znajomość najważniejszych funkcji zdefiniowanych w:
\begin{itemize}
	\item winsock2.h
	\item Ws2bth.h
	\item BluetoothAPIs.h
\end{itemize}
\subsection{Ogólnie pojęcie o mechanizmach rejestracji funkcji callbackowych}
\subsection{Zapoznanie się ze specyfikacją komunikacji poprzez BT}
\subsection{Protokół transferu plików OBEX}
Znajomość poleceń:
\begin{itemize}
	\item CONNECT
	\item PUT
	\item DISCONNECT
\end{itemize}
\section{Zadania do wykonania}
napisać aplikację graficzną, która:
\begin{enumerate}
	\item Wykryć adaptery BT podłączone do PC.
	\item Użyć wybranego adaptera do zdalnego wyszukiwania urządzeń BT.
	\item Pobrać adres MAC wybranego (wyszukanego w pkt. 2) urządzenia.
	\item Dokonać autoryzacji obu urządzeń:
	      \begin{itemize}
		      \item po stronie urządzenia BT autoryzować PC
		      \item po stronie PC autoryzować urządzenie BT.
	      \end{itemize}
	\item Uruchomić urządzenie BT w tryb pracy transferu plików.
	\item Przesłać plik tekstowy do urządzenia BT.
	\item Przesłać plik graficzny do urządzenia BT.
\end{enumerate}
\section{Wnioski}
\end{document}